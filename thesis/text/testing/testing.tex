\chapter{Development Tooling and Testing}

\section{Testing and development}

During the development of the type-level mathematical expression evaluator, several invaluable tools were discovered and utilised that significantly contributed to the implementation. This section is devoted to discussing these tools and their impact on the overall development process.

The core of the development experience is underpinned by TypeScript Standalone Server, also known as \code{tsserver}. \code{tsserver} encapsulates both the compiler and the accompanying language services for use in editors and \acrshort{ide}s, communicating via \acrshort{lsp} to add support for code completion, auto-importing, symbol renaming etc. \code{tsserver} also provides the ability to see the inferred types of any symbol by hovering on top the symbol. This service is invaluable when developing a type-level library, as it allows the developer to break down complex types into smaller pieces, achieving better readability.

Pretty TypeScript Errors \cite{balasianoPrettyTypeScriptErrors2023} attempts to parse and reformat the TypeScript error messages to be more human-readable. This is especially helpful when dealing with complex object types, where the error messages can become unreadable since the error message and the serialised type is printed out on a single line.

\todo{Add an example for Pretty TypeScript Errors}

Another critical tool used when developing the implementation is \code{vscode-twoslash-plugin} \cite{theroxVscodetwoslashqueries2023}.

\todo{Add an example for vscode-twoslash-plugin}

\begin{itemize}
  \item \code{tsserver} allows us to see the inferred types during development by hovering on top of a symbol
  \item \code{vscode-twoslash-plugin} allows us to see the inferred types of a symbol selected by the caret inlined directly in the editor
  \item Testing of types is done by \code{eslint}, \code{@typescript-eslint} and \code{eslint-plugin-expect-type}.
  \item Explanation of \code{\$ExpectType} and \vcode{// ^?} operator
\end{itemize}

\section{CI/CD workflow and release management}

\begin{itemize}
  \item Using Github Actions
  \item Two main workflows: one for running tests, a second one for releasing
  \item Changesets \cite{ChangesetsChangesets2023} - how the package is released
\end{itemize}

\section{Performance testing}

\begin{itemize}
  \item Difference between operations, execution time + number of created types \code{yarn tsc --noEmit --incremental false --generateTrace trace --extendedDiagnostics --listFiles -p tsconfig.json --explainFiles}
  \item Comparison between other libraries (arielhs/ts-arithmetic)
\end{itemize}
