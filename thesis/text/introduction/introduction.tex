\chapter{Introduction}
% \chapter*{Introduction}\addcontentsline{toc}{chapter}{Introduction}\markboth{Introduction}{Introduction}

\section{Motivation}

TypeScript, a typed superset of JavaScript, is quickly gaining popularity in the JavaScript development ecosystem, and type-safety, the concept of validating data types, is \say{eating the world}\cite{jsworldconferenceFredSchottTypesafety2023}. As of 2023, over 66\% of frontend developers are using TypeScript most of the time, either avoiding JavaScript entirely or spending the majority of time working with TypeScript codebases \cite{StateJS2022}. Over the years, TypeScript has transformed from a simple type annotation tool to a full-fledged programming language within the type system itself. Multiple libraries have emerged with advanced TypeScript types to improve the developer experience, for example, Prisma for database type-safety \cite{PrismaPrismaNextgeneration}, Zod for combining schema validation and static type inference \cite{mcdonnellZod2023}, or tRPC for API end-to-end type-safety across boundaries \cite{TRPC2023}. With intelligent suggestions in the editor of choice, TypeScript ensures high code quality while avoiding any runtime costs due to the type system being evaluated during compilation. With editors and \acrshort{ide}s using a language server powered by \acrfull{lsp} to provide the developer with valuable suggestions, there is an incentive to utilise the type system instead of running a daemon alongside or adding a build step.

However, TypeScript is only as powerful as the types declared and received. A significant burden is laid on the maintainers of libraries to provide descriptive and valuable types. This thesis aims to lay out and highlight the capabilities and techniques of the TypeScript type system when applied to a non-trivial problem domain. The type-only implementation of the mathematical expression evaluator serves as a practical case study, demonstrating the power of the TypeScript type system and the benefits of type safety.

\section{What is a static type system}

For years, type systems in programming languages have been a well-known and heavily discussed topic. The main goal of a type system is to provide a formal specification of the types of data that a program can manipulate.

In statically typed languages, the type of a variable is known at compile time. The compiler uses the additional information about data types to verify the source code during compilation. The data type itself can be deduced from the usage in the code (type inference), or a programmer explicitly specifies the data type of a variable before usage. Examples of such languages using static typing are, for instance, Java, C\# or C++.

Whereas in dynamically typed languages, the type of a variable is determined at runtime based on the value being assigned, and it does not need to be explicitly declared by the developer or known at compile time via type inference. Some of the popular dynamically typed languages include Python, Ruby, PHP, and, most notably, JavaScript, which is widely used to create interactive and dynamic user interfaces on the web platform. Dynamically typed languages tend to be more flexible and allow developers, especially beginner developers, to write code faster and iterate quicker.

On the other hand, static typing offers numerous compelling benefits that can enhance the development process. First, a large class of errors is caught earlier in the process. This reduces the likelihood of bugs and runtime issues that can be difficult to diagnose and debug. With static typing, developers can rely on a compiler system to ensure the code conforms to the expected data types. Developers can also refactor existing typed code more confidently, as the system gives developers direct feedback when refactoring.

Furthermore, by writing type annotations, developers are actively self-documenting the code, making it more readable and easier to understand, especially when dealing with unfamiliar code. Finally, even though an initial commitment is necessary by writing type annotations at first, a more powerful type system can determine the developer's intent without writing additional code as the development progresses.

\section{Structure of the work}

This thesis will provide a comprehensive analysis of relevant advanced constructs found in the TypeScript type system and how they can be used to allow robust meta-programming within the types themselves. An implementation of a generic math expression evaluator library that operates strictly on the type level is provided to demonstrate the capabilities of the type system, followed by a discussion on testing and performance of the library and the impact on type-checking and development experience in the editor.
