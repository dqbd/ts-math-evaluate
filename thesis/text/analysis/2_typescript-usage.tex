\section{Usage of TypeScript}

The TypeScript project is made of two major parts available to developers:

\begin{itemize}
  \item \code{tsc}: The TypeScript Compiler, which is responsible for both type-checking and outputting valid JavaScript files,
  \item \code{tsserver}: The TypeScript Standalone Server, which encapsulates the TypeScript Compiler and language services for use in editors and \acrshort{ide}s \cite{StandaloneServerTsserver}.
\end{itemize}

While a type-checker is most likely executed manually more often and is the entry point for developers when using TypeScript, the language server is equally as useful, as it communicates with the editor via \acrfull{lsp} to provide important language services. These include code completion, auto-importing, symbol renaming etc.

The term \say{compilation} in this thesis refers specifically to the process of type erasure itself. Although the source code may contain various type-related errors, the TypeScript Compiler (\code{tsc}) will generate valid JavaScript files by default as long as the input source file can be correctly parsed. This enables developers to gradually improve their code and quickly iterate on its functionality without fixing type errors immediately. In this sense, the TypeScript Compiler functions more like a code analyser rather than a traditional compiler seen in other programming languages. Regardless, in this thesis, the terms \say{compiling} and \say{type-checking} will be used interchangeably.