\section{Prior Art}

There are multiple basic implementations of math operations in TypeScript. Tasks regarding basic math operations are even part of the TypeChallenges collection \cite{TypechallengesTypechallenges2023}. However, most of them only work on integers, as they work on tuple expansion, which will be further discussed in the implementation part of this thesis.

Nevertheless, multiple libraries in the \acrshort{npm} registry provide basic math calculations within the TypeScript type system, but none provide a fully typed parser of mathematical expressions. Some of the libraries found do provide type utilities that operate on floating-point numbers instead of integers, such as \code{type-fest} \cite{sorhusSindresorhusTypefest2023} or \code{typescript-lodash} \cite{kawayilinlinKawayiLinLinTypescriptlodash2023}. The most comprehensive implementation of math operations can be found in the \code{ts-arithmetic} library \cite{arielTypeLevelArithmetic2023}, which provides a fully typed implementation of division.

% \begin{itemize}
%   \item \code{kawayiLinLin/typescript-lodash} - floating point numbers, addition, subtraction, division by two
%   \item \code{arielhs/ts-arithmetic} - floating point numbers, addition, subtraction, multiplication, division, mod, negate, abs, compare, max, min, sign
%   \item \code{type-fest} - parsing float, sign modification
%   \item \code{hotscript} - integers, addition, subtraction, multiplication, division, mod, negate, abs, compare, max, min, sign
% \end{itemize}