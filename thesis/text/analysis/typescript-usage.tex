\section{Usage of TypeScript}

The TypeScript project is made of two major parts available to developers:

\begin{itemize}
  \item \code{tsc}: the TypeScript Compiler, which is responsible for both type checking and outputting valid JavaScript files,
  \item \code{tsserver}: the TypeScript Standalone Server, which encapsulates the TypeScript Compiler and language services for use in editors and IDEs \cite{StandaloneServerTsserver}.
\end{itemize}

While a type-checker is most likely executed manually more often and is the entry point for developers when using TypeScript, the language server is equally as useful, as it communicates with the editor via Language Server Protocol (LSP) to provide important language services. These include code completion, auto-importing, symbol renaming etc.

Unlike in the other languages, the compilation step itself is understood to only mean the type erasure itself. Even though the source code itself can have various type errors, \code{tsc} will still, by default, emit JavaScript files as long as the input source file can be parsed by both the scanner and the parser. This allows developers to progressively update their code and iterate quickly on the functionality without immediately dealing with the type errors, acting more as a linter than a compiler. Regardless, in this thesis, \say{compiling} and \say{type-checking} will be used interchangeably.
