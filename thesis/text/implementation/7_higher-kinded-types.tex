\section{Higher-kinded types}\label{sec:higher-kinded-types}

\acrfull{hkt}, also known as higher-order types, are a powerful type system language feature that enables describing expressive generic types by allowing accepting other generic types as type arguments. To demonstrate, consider the following Listing \ref{lst:duplicate-generic-types}. As it can be seen, all three generic types do essentially the same type instantiation, only with different type constructors.

\begin{listing}[ht]
  \begin{minted}{TypeScript}
type Foo<O> = O extends string ? `Foo<${O}>` : never
type Bar<O> = O extends string ? `Bar<${O}>` : never
type Baz<O> = O extends string ? `Baz<${O}>` : never

type MapValuesWithFoo<O> = { [K in keyof O]: Foo<O[K]> }
type MapValuesWithBar<O> = { [K in keyof O]: Bar<O[K]> }
type MapValuesWithBaz<O> = { [K in keyof O]: Baz<O[K]> }
\end{minted}
  \caption{Duplicate generic types}\label{lst:duplicate-generic-types}
\end{listing}

With \acrshort{hkt}s, it is possible to define a single higher-order generic type that accepts a type constructor as an argument. The type constructor is then applied to each property of the object type. The result is shown in Listing \ref{lst:proposed-hkt-syntax}.

\begin{listing}[ht]
  \begin{minted}{TypeScript}
type MapValuesWith<O, T<~>> = { [K in keyof O]: T<O[K]> }

type MapValuesWithFoo<O> = MapValuesWith<O, Foo>;
type MapValuesWithBar<O> = MapValuesWith<O, Bar>;
type MapValuesWithBaz<O> = MapValuesWith<O, Baz>;
\end{minted}
  \caption{Proposed HKT syntax in TypeScript}\label{lst:proposed-hkt-syntax}
\end{listing}

With higher-kinded types, it is possible to declare a monad type \cite{wadlerMonadsFunctionalProgramming1993} or applicative functors \cite{mcbrideApplicativeProgrammingEffects2008}, design patterns commonly found in functional programming languages such as Haskell or Scala. However, as of writing, higher-kinded types are not natively supported by TypeScript \cite{DocumentationTypeScriptFunctional}. Fortunately, it is possible to emulate the behaviour of higher-kinded types.

There are two ways to achieve the behaviour of \acrshort{hkt}. One such way can be achieved by implementing lightweight higher-kinded polymorphism \cite{yallopLightweightHigherKindedPolymorphism2014} and defunctionalisation of kinds \cite{reynoldsDefinitionalInterpretersHigherorder1972}, a technique for translating higher-order programs into a first-order language. The main idea is to create a mapping of unique names of type constructors to their implementations. Afterwards, a \code{Kind} utility converts the name and the appropriate type argument to the corresponding higher-kinded type. An example can be seen in Listing \ref{lst:emulating-hkt-legacy}.

\begin{listing}[ht]
  \begin{minted}{TypeScript}
type URItoKind<A> = { "Foo": Foo<A>; "Bar": Bar<A>; "Baz": Baz<A> }
type URI = keyof URItoKind<unknown>
type Kind<F extends URI, A> = URItoKind<A>[F];

type MapValuesWith<O, Type extends URI> = Kind<Type, O>
\end{minted}
  \caption{\acrshort{hkt} emulation using lightweight higher-kinded polymorphism}\label{lst:emulating-hkt-legacy}
\end{listing}

This method is historically used in libraries for typed functional programming such as \code{fp-ts} \cite{GcantiFptsFunctional}. Unfortunately, this method requires a central registry of URIs that are used to identify the appropriate type constructor and extendability based on module augmentation is limited.

The other possible method for implementing \acrshort{hkt}s is by utilising the properties of type intersection with \code{this}. The interface \code{Fn} provides a generic template for a callable function. Each such callable function must extend from \code{Fn} and depend on \code{this["input"]} to instantiate the type of \code{output} key. The \code{Call} generic type accepts such a function as a type parameter \code{Function} alongside the input as \code{Input} type parameter. Finally, the type constructor of \code{Call} will intersect the provided \code{Function} with the object type wrapping the \code{Input} type parameter, essentially providing the function with the desired arguments. The created \code{output} key can be extracted using indexed access types. This method is thoroughly used in HOTscript \cite{vergnaudHigherOrderTypeScriptHOTScript2023}, and a simplified implementation can be seen in Listing \ref{lst:emulating-hkt}.

\clearpage

\begin{listing}[ht]
  \begin{minted}{TypeScript}
interface Fn { input: unknown; output: unknown; }

type Call<Function extends Fn, Input> = (Function & { input: Input })["output"];

interface Foo extends Fn {
  output: this["input"] extends infer O extends string ? `Foo<${O}>` : never;
}

interface Bar extends Fn {
  output: this["input"] extends infer O extends string ? `Bar<${O}>` : never;
}

interface Baz extends Fn {
  output: this["input"] extends infer O extends string ? `Baz<${O}>` : never;
}

type MapValuesWith<O, Wrap extends Fn> = {
  [K in keyof O]: Call<Wrap, O[K]>
}
\end{minted}
  \caption{Type intersection for emulating \acrshort{hkt}s}\label{lst:emulating-hkt}
\end{listing}

The most popular implementation of the latter method, HOTScript, exposes most of the core functionality of the library as a public-facing API. Thus, an additional public-facing API for mathematical operations has been exposed for users of HOTScript, extending the library with an advanced mathematical expression evaluator implemented in this work.
